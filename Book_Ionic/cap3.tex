\chapter{Usuario ingeniero}
Se trata de usar un esquema de men\'us sencillo parecido a un panel de control  y la personalizaci\'on de men\'u lateral como usan las apps de gmail para adaptarlo a cada usuario .\\
Navigation view :En el men\'u cuando demos click en una determinada opci\'on se ejecutar\'a una determinada acci\'on 
Si se da click en la esquina   superior izquierda aparece el Navigation view .\\
Se aprecia que aparece el nombre real del usuario logueado en el navigation view.
\section{Funcionalidades de usuario}
Esta interfaz es como un panel de control para el usuario ingeneniero con funcionalidades personalizadas, adem\'as de la exposici\'on de sus datos registrados ,actualizaci\'on de su posici\'on actual,de su disponibilidad,email,celular entre otros datos.
\subsection{Men\'u de bienvenida}
En esta secci\'on nos encontramos la exposici\'on de los datos que no pueden ser modificados,estos datos son los relacionados al ingeniero que se logueo exitosamente.
Los datos exhibidos son : Nombre del ingeniero, C.I.P., fecha de categorizaci\'on,universidad,especialidad,dni,estado y categoria.
