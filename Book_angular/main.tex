\documentclass[12pt,fleqn,x11names,table]{book}                                             
\usepackage[margin=1in]{geometry}% Dimensiones y márgenes=1 pulgada alrededor del texto  
\usepackage{amsmath,amssymb,amsfonts,latexsym,cancel,xcolor,color}
  \usepackage[english,spanish,es-tabla]{babel}
\usepackage{graphicx}
\usepackage{setspace}%%Para dar interlineado
\usepackage{float}%% Para usar H en figuras
\usepackage{multirow} % para las tablas
\usepackage{newcent} %New Century Schoolbook
\usepackage{verbments}%Requiere Python, Pygments y Habilitar en las opdenes de pdflatex lo sihuiente
\usepackage{tcolorbox, empheq}%Para cajas de ejercicios
\input{EstiloB}% = Paquetes y código de diseño "B"
%%Colores para cajas de ejercicios
\definecolor{azulF}{rgb}{.0,.0,.3} % Azul
\definecolor{rojoF}{RGB}{212,0,0} % Rojo
\newcommand{\Z}{\mathbb{Z}} % \Z
%%Estilo para cajas de ejercicios
\tcbuselibrary{skins,breakable,listings,theorems}
\tcbset{opteqA/.style={%
		tcboxraise base,
		nobeforeafter,
		extrude by=-2mm,
		colback=red!50!black!20,
		colframe=red!50!black!20}}
%%Colores  para cajas de programa
\definecolor{verbmentsbgcolor}{rgb}{0.9764, 0.9764, 0.9762}
\definecolor{verbmentscaptionbgcolor}{rgb}{0.1647, 0.4980, 1}	
% Insertar portada
\usepackage{pdfpages}
\begin{document}
	\input caratula.tex		%% Caratula
	\spacing{1.2}			%% espaciado
	\tableofcontents		%%Indice
	\listoffigures			%%Indice figuras
	%%\listoftables			%%Indice tablas
	\thispagestyle{empty}
	\mainmatter			    %empieza la numeración de las páginas    
	\cleardoublepage
	\pagestyle{fancy}                      % Habilitar encabezados
	\pagenumbering{arabic}                 % Numeración arabiga
	\ansj=1                                % Cap 1 inicializa listas      
	\input listaProgramasVB.tex
	\input cap1.tex
	\input cap2.tex
	\input cap3.tex
\end{document}


