\chapter{Preguntas tecnicas FullStack}
\section{TDD}
Es el desarrollo en base a pruebas.
\section{Ajax}
Tecnologia para hacer peticiones de manera asincrona.
\section{div vs span}
El div es un bloque donde se puede poner otros bloques
en span esta limitado a elementos en una linea
\section{Beneficios de SOLID}
Codigo mantenible,facil de escalarlo con el tiempo,desacoplado,buenas practicas.
\section{Verbos HTTP}
POST Envio de informacion
PUT Actualizacion total de un recurso
GET Recuperar informacion
DELETE Para eliminar un recurso
PACTH Actualizacion parcial de un recurso
OPTION Para otro tipo de operaciones
\section{Codigos de respuesta HTTP}
1XX
2XX Operacion exitosa
3XX
4XX Error por parte del cliente
5XX Error por parte del servidor
\section{Como despliegas una aplicacion fullstack}
De forma independiente se puede desplegar el backend y frontend.
El backend en un servidor tradicional Jboss,Wildfly,usar un contenedor  y subirlo a un proveedor cloud.
El frontend  genera archivos compilados: css,js,html y estos pueden subirse a un apache,s3(ngnix).
Obviamente hay un tema de arquitectura.
\section{SQL VS NoSQL}
Si el app su prioridad los datos necesitan tener estructura y respetar los principios ACID
entonces BD relacional (SQL) .En cambio si se quiere escalabilidad,flexibilidad en la extructura,busqueda rapida se recomienda BD no relacionales ()NOSQL).
\section{Diferencia entre REST Y SOAP}
Rest es una arquitectura en cambio SOAP es un protocolo que necesita un contrato(WDSL).\\
Rest es mas sencillo de usar.
\section{Diferencias LET,CONST,VAR}
VAR :Esta obsoleto.
LET :Define una variable en el  alcance donde fue declarada esa variable
CONST : No se puede reasignar un valor.
\section{Funcion flecha}
Una forma abreviada de declarar una funcion.Una funcion anonima.
\section{Normalizacion de datos}
Tecnica que se usa para reducir la redundancia.Mantener la integridad.
\section{Como proteger los servicios expuestos por el backend}
Protocolo u2,el header con un identificador ajeno a JWT.HTTPS.
\section{Utilidad de JWT}
Un mecanismo que tiene un formato especial,que al decodificarlo presenta una firma,payload,
asignasure,header.Es un identificador.
\section{OAUTH2}
Un protocolo de autorizacion para acceder a ciertos permisos en otras aplicaciones.
\section{Gestores o productos de seguridad}
WSO2 Identity server.\\
Key cloack.\\
AWS Cognito.
\section{Se puede modificar el payload de un JWT}
Si ,pero la firma lo detectara como corrupto.
\section{Angular VS React}
Caundo hay mucha manipulacion o demanda de la pagina,react tiene mejores tiempos de respuesta.
En cambio angular tiene las tecnologias listas para usar,standares y personal tecnologico mas
disponible.
\section{Como proteges los aplicativos en el frontend}
Mediante accesos no autorizados.\\
Proteger la autorizacion.Acceder a un recurso que no me pertenece.\\
Validar la caducidad de los tokens.
\section{Programacion re4activa}
Procesos asincronos,mejores tiempos de respuesta,orientado a eventos.
\section{Que es un MOCK}
Objetos ya programados,con una salida esperada.Datos ficticios.Usado en testing.
\section{Inyeccion de dependencias}
Es un patron de diseno.La misma tecnologia es el encargado de proporcionar las instancias  en el tiempo oportuno.Permite desacoplar el codigo.
\section{Preprocesadores de css conoces.}
less\\
Sas\\
Styles.
\section{UI VS UIX}
UI es la interfaz de usuario.Mientras que UIX se refiere a la experiencia de usuario,la usabilidad.
\section{Que es COORS}

\section{CSFR}
Ataque se envia codigo malicioso al servidor.Ataca vulneravilidades del cliente para explotar un servicio
\section{Evitar inyeccion SQL}
La misma dependencia de JPA prevee esto.Usando procedimientos almacenados
\section{Que es una interface en Typescript}
Define una estructura,como una plantilla.
\section{Especificacion de Javascript}
Ecmascript .Apartir de Ecmascript 6 aparecio la definicion de let y const.
Tipo de datos de Javascript:String,booleano,objeto e indefinido.
\section{Que hace el operador delete de JS}
Elimina una propiedad de un objeto.
\section{Storage usado en navegadores}
Local storage,su alcance a pesar de cerrar session continua.\\
Session storage ,su alcance es mas temporal.
La diferencia entre ambos es el alcance.
\section{DOM}
Es el arbol de nodos de las etiquetas HTML.
\section{Callback}
Funcion que tiene paramtro otra funcion.
\section{Callback Hell}
Se anidan varios callback.
\section{Que es un observable}
Es un flujo de datos,para que cualquier suscripcion se entere si ha existido un cambio.
\section{Landinpage usarias jquery o angular}
Landinpage si es pequeño se usa jquery, si el Landinpage va a crecer o se va a reutilizar entonces angular.
\section{Nivel junior}
1.	¿Cuál es la diferencia entre error y excepción?\\
2.	Diferencia entre finally y finalize\\
3.	¿Para qué sirve la palabra reservada final?\\
4.	¿Cuáles son los 4 pilares de la POO?\\
5.	¿Cuál es la diferencia entre una interfaz y una clase abstracta?\\
6.	¿Qué es un Servlet?
\section{Nivel Semi Senior}
1.	¿Qué es el autoboxing?\\
2.	Diferencia entre equals e ==\\
3.	Define agregación y su diferencia contra composición\\
4.	¿Qué tipos de excepciones existen?\\
5.	¿Cuál es la diferencia entre overload y override?\\
6.	¿Qué es un deadlock?
\section{Nivel Senior}
1.	¿Para qué sirve transient?\\
2.	Explica el ciclo de vida de un servlet\\
3.	¿Cuál es la diferencia entre hilo y proceso?\\
4.	Explica el ciclo de vida de un hilo\\
5.	¿Cómo evitas un deadlock?\\
6.	¿Cómo se usa la palabra reservada synchronized?\\
7.	¿Cuál es la diferencia entre los métodos sleep y await en hilos?
\section{Preguntas de Base de datos}
1.	¿Para qué sirven los índices?\\
2.	¿Cuál es la diferencia entre where y having?\\
3.	Diferencia entre procedimiento almacenado y función\\
4.	Formas de unir dos tablas de una base de datos (joins)\\
5.	¿Qué es la normalización?
\section{Preguntas de Middleware}
1.	¿Qué métodos de HTTP existen?\\
2.	¿Qué es REST?\\
3.	¿Cuál es la diferencia entre REST y SOAP?\\
4.	¿Para qué sirve qualified? (realmente es pregunta sobre singleton)\\
5.	¿Cómo se usa la anotación value?\\
6.	¿Qué es la inversión de control?\\
7.	¿Qué es la inyección de dependencias?\\
8.	Explica que es el contenedor IOC de spring\\
9.	¿Qué anotaciones se usan en spring para inyección de dependencias?\\
10.	Diferencia entre spring y spring boot\\
11.	Diferencia entre Repository, Service y Controller\\
12.	Formas de inyectar dependencias en Spring\\
13.	¿Cuál es el ciclo de vida de spring?
